
% This LaTeX was auto-generated from MATLAB code.
% To make changes, update the MATLAB code and republish this document.

\documentclass{article}
\usepackage{graphicx}
\usepackage{color}

\sloppy
\definecolor{lightgray}{gray}{0.5}
\setlength{\parindent}{0pt}

 \input{preamble} \begin{document} \maketitle \tableofcontents 

    
    
{}


 


\section{Problem 20.2-1}

\begin{par}

Solute A is diffusing at unsteady state into a semi-infinite medium of
pure B and undergoes a first-order reaction with B. Solute A is
dilute. Calculate the concentration $c_A$ at points z = 0, 4, and
$10~\textup{mm}$ from the surface for $t=1\times 10^5~\textup{s}$.
Physical property data are $D_{AB}=1\times
10^{-9}~\textup{m}^2/\textup{s}$, $k'=1\times 10^{-4}~\textup{s}^{-1}$,
$c_{A0} = 1.0~\textup{kg mol/m}^3$. Also calculate the
$\textup{kg mol absorbed/m}^2$.
The concentration of A at point z is given by \vref{conc_empirical_1}
(corrected version),
\begin{equation} \label{conc_empirical_1}
\begin{split}
\frac{C_A}{C_{A0}}=&\frac{1}{2}\exp\left(-z\sqrt{k'/D_{AB}}\right)
 * \textup{erfc}\left(\frac{z}{2\sqrt{tD_{AB}}}-\sqrt{k't}\right) \\
 &+ \frac{1}{2}\exp\left(z\sqrt{k'/D_{AB}}\right)
 * \textup{erfc}\left(\frac{z}{2\sqrt{tD_{AB}}}+\sqrt{k't}\right)
\end{split}
\end{equation}
\Vref{conc_empirical_1} was incorrect in the textbook, and has
been modified according to the instructor's email.

\end{par} \vspace{1em}
\begin{verbatim}
u = symunit;
t = 1e5 * u.s;
D_AB = 1e-9 * u.m^2 / u.s;
k_prime = 1e-4 * u.s^-1;
C_A0 = 1.0 * (u.kg * u.mol) / (u.m^3);
C_A = @(z) C_A0 * ...
    (0.5*exp(-z*sqrt(k_prime/D_AB))*erfc((z/(2*sqrt(t*D_AB))) - sqrt(k_prime*t)) + ...
    0.5*(exp(z*sqrt(k_prime/D_AB)))*erfc((z/(2*sqrt(t*D_AB))) + sqrt(k_prime*t)));
disp(unitString(C_A(0), 'C_A (0 mm)'))
disp(unitString(C_A(0.004 * u.m), 'C_A (4 mm)'))
disp(unitString(C_A(0.010 * u.m), 'C_A (10 mm)'))
\end{verbatim}

        \color{lightgray} \begin{verbatim}C_A (0 mm): 1 (kg*mol)/m^3
C_A (4 mm): 0.28226 (kg*mol)/m^3
C_A (10 mm): 0.042328 (kg*mol)/m^3
\end{verbatim} \color{black}
    \begin{par}

The amount of A absorbed per square meter is given by
\vref{conc_empirical_2} (corrected version),
\begin{equation} \label{conc_empirical_2}
Q=C_{A0}\sqrt{D_{AB}/k'}\left[\left(k't+\frac{1}{2}\right)\textup{erf}\sqrt{k't}+\sqrt{k't/\pi} e^{-k't}\right]
\end{equation}

\end{par} \vspace{1em}
\begin{verbatim}
Q = C_A0*sqrt(D_AB/k_prime)*((k_prime*t+0.5)*erf(sqrt(k_prime*t) + ...
    sqrt((k_prime*t)/pi)*exp(-k_prime*t)));
disp(unitString(Q))
\end{verbatim}

        \color{lightgray} \begin{verbatim}Q: 0.033204 (kg*mol)/m^2
\end{verbatim} \color{black}
    

\section{Problem 21.1-2}

\begin{par}
Prove or show the following relationships, starting with the flux equations:
\end{par} \vspace{1em}
\begin{par}

\subsection{Part a}
Convert $k_c'$ to $k_y$ and $k_G$.

\end{par} \vspace{1em}
\begin{par}
The flux equations involving these coefficients are, $$ N_A = k_c(C_{A1}-C_{A2}) = k_G(p_{A1}-p_{A2}) = k_y(y_{A1}-y_{A2}) $$ the conversion between $k_c'$ and $k_c$ is given in the flux equation for A diffusing through stagnant, non-diffusing B,
\end{par} \vspace{1em}
\begin{par}

\[N_A = \frac{k_c'}{x_{BM}}\left(C_{A1}-C_{A2}\right) =
k_c\left(C_{A1}-C_{A2}\right)\]
\[ k_c = \frac{k_c'}{x_{BM}} = \frac{k_c' P}{p_{BM}} \]
\[ k_y = \frac{k_c' P}{p_{BM}}\cdot\frac{C_{A1}-C_{A2}}{y_{A1}-y_{A2}} = \frac{k_c' P}{RT\cdot y_{BM}} \]
\[ k_G = \frac{k_c' P}{p_{BM}}\cdot\frac{C_{A1}-C_{A2}}{p_{A1}-p_{A2}} = \frac{k_c' P}{RT\cdot p_{BM}} \]

\end{par} \vspace{1em}
\begin{par}

\subsection{Part b}
Convert $k_L$ to $k_x$ and $k_x'$.

\end{par} \vspace{1em}
\begin{par}
The flux equations involving these coefficients are, $$ N_A = k_L(c_{A1}-c_{A2}) = k_x(x_{A1}-x_{A2}) $$ A relationship for $k_x$ can be found by substituting $x=c_{A}/c$, $$ k_x = k_L\frac{c_{A1}-c_{A2}}{x_{A1}-x_{A2}} = k_L\frac{c_{A1}-c_{A2}}{\frac{c_{A1}}{c}-\frac{c_{A2}}{c}} = k_L\cdot c $$ $$ k_x' = k_x\cdot x_{BM} = k_L \cdot c \cdot x_{BM} $$
\end{par} \vspace{1em}
\begin{par}

\subsection{Part c}
Convert $k_G$ to $k_y$ and $k_c$.

\end{par} \vspace{1em}
\begin{par}
The flux equations involving these coefficients are, $$ N_A = k_G(p_{A1}-p_{A2}) = k_y(y_{A1}-y_{A2}) = k_c(C_{A1}-C_{A2}) $$ By substituting $p_A=y_A P$, $$ k_y = k_G\frac{p_{A1}-p_{A2}}{y_{A1}-y_{A2}} = k_G\frac{y_{A1}P-y_{A2}P}{y_{A1}-y_{A2}} = k_G\cdot P $$ Finally, substituting $C=P/RT$, $$ k_c = k_G\frac{p_{A1}-p_{A2}}{C_{A1}-C_{A2}} = k_G\frac{p_{A1}-p_{A2}}{(p_{A1}-p_{A2})/RT} = k_G \cdot RT $$
\end{par} \vspace{1em}


\section{Problem 21.1-3}

\begin{par}
In a wetted-wall tower an air- $\textup{H}_2\textup{S}$ mixture is flowing by a film of water that is flowing as a thin film down a vertical plate. The $\textup{H}_2\textup{S}$ is being absorbed from the air to the water at a total pressure of $1.50~\textup{atm abs}$ and $30~\textup{C}$. A value for $k_c'$ of $9.567\times 10^{-4}~\textup{m/s}$ has been predicted for the gas-phase mass-transfer coefficient. At a given point, the mole fraction of $\textup{H}_2\textup{S}$ in the liquid at the liquid-gas interface is $2.0(10^{-5})$ and $p_A$ of $\textup{H}_2\textup{S}$ in the gas is $0.05~\textup{atm}$. The Henry's law equilibrium relation is $p_A(\textup{atm}) = 609x_A$ (mole fraction in liquid). Calculate the rate of absorption of $\textup{H}_2\textup{S}$. (\textit{Hint}: Call point 1 the interface and point 2 the gas phase. Then, calculate $p_{A1}$ from Henry's law and the given $x_A$. The value of $p_{A2}$ is $0.05~\textup{atm}$.)
\end{par} \vspace{1em}
\begin{par}
The mass transfer coefficient given is $k_c'$, which can be converted to $k_G$ for a flux equation based on partial pressures. $$ k_G = \frac{k_c' P}{RT\cdot p_{BM}} $$ $$ N_A = k_G(p_{A1} - p_{A2}) $$
\end{par} \vspace{1em}
\begin{verbatim}
u = symunit;
P = 1.5 * u.atm;
T = rewrite(30 * u.Celsius, u.K, 'Temperature', 'absolute');
k_c_prime = 9.567e-4 * u.m / u.s;
x_A1 = 2.0e-5;
p_A1 = 609 * x_A1 * u.atm;
p_A2 = 0.05 * u.atm;
p1 = separateUnits(p_A1); p2 = separateUnits(p_A2); p = separateUnits(P);
p_BM = ((p-p1)-(p-p2))/log((p-p1)/(p-p2)) * u.atm;
R = 8.2057338e-5 * (u.m^3 * u.atm) / (u.mol * u.K);
k_G = (k_c_prime * P) / (R * T * p_BM);
N_A = k_G * (p_A1 - p_A2) * 1e-3 * u.kg;  % convert to kg-mol
disp(unitString(N_A))
\end{verbatim}

        \color{lightgray} \begin{verbatim}N_A: -1.4854e-06 (kg*mol)/(m^2*s)
\end{verbatim} \color{black}
    

\section{Problem 21.2-1}

\begin{par}
A fluid is flowing in a vertical pipe and mass tranfer is occurring from the pipe wall to the fluid. Relate the convective mass-transfer coefficient $k_c'$ to the variables $D$, $\rho$, $\mu$, $v$, $D_{AB}$, $g$, and $\Delta{\rho}$, where $D$ is pipe diameter, $L$ is pipe length, and $\Delta{\rho}$ is the density difference.
\end{par} \vspace{1em}
\begin{par}
According to Buckingham's pi theorem, given nine independent variables in three physical dimensions, there are six $\pi-$ groups to construct.
\end{par} \vspace{1em}
\begin{par}
Vector order: [Length Mass time]
\end{par} \vspace{1em}
\begin{verbatim}
D = [1 0 0];
L = [1 0 0];
rho = [-3 1 0];
mu = [-1 1 -1];
v = [1 0 -1];
D_AB = [2 0 -1];
g = [1 0 -2];
delta_rho = [-3 1 0];
k_c_prime = [1 0 -1];

pi_1 = ([D' rho' mu']\-k_c_prime')'
\end{verbatim}

        \color{lightgray} \begin{verbatim}
pi_1 =

     1     1    -1

\end{verbatim} \color{black}
    \begin{par}
$\pi_1 = \frac{D \rho k_c'}{\mu}$
\end{par} \vspace{1em}
\begin{verbatim}
pi_2 = ([D' rho' mu']\-v')'
\end{verbatim}

        \color{lightgray} \begin{verbatim}
pi_2 =

     1     1    -1

\end{verbatim} \color{black}
    \begin{par}
$\pi_2 = \frac{D \rho v}{\mu}$
\end{par} \vspace{1em}
\begin{verbatim}
pi_3 = ([D' rho' mu']\-D_AB')'
\end{verbatim}

        \color{lightgray} \begin{verbatim}
pi_3 =

     0     1    -1

\end{verbatim} \color{black}
    \begin{par}
$\pi_3 = \frac{\rho D_{AB}}{\mu}$
\end{par} \vspace{1em}
\begin{verbatim}
pi_4 = ([D' rho' mu']\-g')'
\end{verbatim}

        \color{lightgray} \begin{verbatim}
pi_4 =

     3     2    -2

\end{verbatim} \color{black}
    \begin{par}
$\pi_4 = \frac{D^3 \rho^2 g}{\mu^2}$
\end{par} \vspace{1em}
\begin{verbatim}
pi_5 = ([D' rho' mu']\-delta_rho')'
\end{verbatim}

        \color{lightgray} \begin{verbatim}
pi_5 =

     0    -1     0

\end{verbatim} \color{black}
    \begin{par}
$\pi_5 = \frac{\Delta\rho}{\rho}$
\end{par} \vspace{1em}
\begin{verbatim}
pi_6 = ([D' rho' mu']\-L')'
\end{verbatim}

        \color{lightgray} \begin{verbatim}
pi_6 =

    -1     0     0

\end{verbatim} \color{black}
    \begin{par}
$\pi_6 = \frac{L}{D}$
\end{par} \vspace{1em}
\begin{par}
These dimensionless groups can be combined to describe the system. This combination was done pencil-and-paper with a lot of frustration, and the results are shown here.
\end{par} \vspace{1em}
\begin{par}

\[ \frac{\pi_1}{\pi_3} = f\left(\pi_4 \pi_5 \pi_6^3, \pi_2,
\pi_3^{-1}\right) \]
\[ \frac{k_c'D}{D_{AB}}=f\left(\frac{gL^3\rho\Delta\rho}{\mu^2},
\frac{Dv\rho}{\mu},\frac{\mu}{\rho D_{AB}}\right) \]

\end{par} \vspace{1em}


\section{Problem 21.3-1}

\begin{par}
Using the data and physical properties of Example 21.3-2, calculate the flux for a water velocity of $0.152~\textup{m/s}$ and a plate length of $L = 0.137~\textup{m}$. Do not assume that $x_{BM}=1.0$ but actually calculate its value.
\end{par} \vspace{1em}
\begin{par}
The Schmidt and Reynolds numbers can be calculated from the given quantities,
\end{par} \vspace{1em}
\begin{verbatim}
u = symunit;
T = rewrite(26.1 * u.Celsius, u.K, 'Temperature', 'absolute');
L = 0.137 * u.m;
v = 0.152 * u.m / u.s;
solubility = 0.02948 * (u.kg * u.mol) / (u.m^3);
D_AB = 1.245e-9 * u.m^2 / u.s;
mu = 8.71e-4 * u.Pa * u.s;
rho = 996 * u.kg / u.m^3;
N_Sc = simplify(mu / (rho * D_AB));
N_Re = simplify((L * v * rho)/mu);
disp(unitString(N_Sc))
disp(unitString(N_Re))
\end{verbatim}

        \color{lightgray} \begin{verbatim}N_Sc: 702.408 1
N_Re: 23812.5189 1
\end{verbatim} \color{black}
    \begin{par}
This value for the Reynolds number corresponds to this equation for mass flux, $$ J_D = 0.99N_{Re,L}^{-0.5} = \frac{k_c'}{v}\left(N_{Sc}\right)^{2/3} $$
\end{par} \vspace{1em}
\begin{verbatim}
syms k_c_prime;
k_c_prime = solve(0.99*N_Re^-0.5 == (k_c_prime/v)*(N_Sc^(2/3)));
FW_water = 18.02* u.kg / (u.kg * u.mol);
c = rho / FW_water;
x_A1 = 0;
x_A2 = solubility / (solubility + c);
x_BM = (1 - (1 - x_A2))/log(1/(1 - x_A2));
k_c = k_c_prime / x_BM;
N_A = k_c * solubility;
disp(unitString(N_A))
\end{verbatim}

        \color{lightgray} \begin{verbatim}N_A: 3.6391e-07 (kg*mol)/(m^2*s)
\end{verbatim} \color{black}
    


 \section{Referenced Functions} \subsection{unitString.m} \begin{verbatim} function displayString = unitString(quantity, name)
%UNITSTRING Display a 1x2 sym with symbolic units
% USAGE: unitString(some_quantity, name)
% OUTPUT:
%   - displayString: char vector containing name, scalar, and units
if nargin < 2
    n = inputname(1);
else
    n = name;
end

[s, U] = separateUnits(quantity);
formatSpec = '%s: %s %s';
displayString = sprintf(formatSpec, n, num2str(double(s)), symunit2str(U));
end \end{verbatim} \end{document}